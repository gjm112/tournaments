% Options for packages loaded elsewhere
\PassOptionsToPackage{unicode}{hyperref}
\PassOptionsToPackage{hyphens}{url}
\PassOptionsToPackage{dvipsnames,svgnames,x11names}{xcolor}
%
\documentclass[
  12pt,
]{article}
\usepackage{amsmath,amssymb}
\usepackage{iftex}
\ifPDFTeX
  \usepackage[T1]{fontenc}
  \usepackage[utf8]{inputenc}
  \usepackage{textcomp} % provide euro and other symbols
\else % if luatex or xetex
  \usepackage{unicode-math} % this also loads fontspec
  \defaultfontfeatures{Scale=MatchLowercase}
  \defaultfontfeatures[\rmfamily]{Ligatures=TeX,Scale=1}
\fi
\usepackage{lmodern}
\ifPDFTeX\else
  % xetex/luatex font selection
\fi
% Use upquote if available, for straight quotes in verbatim environments
\IfFileExists{upquote.sty}{\usepackage{upquote}}{}
\IfFileExists{microtype.sty}{% use microtype if available
  \usepackage[]{microtype}
  \UseMicrotypeSet[protrusion]{basicmath} % disable protrusion for tt fonts
}{}
\makeatletter
\@ifundefined{KOMAClassName}{% if non-KOMA class
  \IfFileExists{parskip.sty}{%
    \usepackage{parskip}
  }{% else
    \setlength{\parindent}{0pt}
    \setlength{\parskip}{6pt plus 2pt minus 1pt}}
}{% if KOMA class
  \KOMAoptions{parskip=half}}
\makeatother
\usepackage{xcolor}
\usepackage[margin=1in]{geometry}
\usepackage{graphicx}
\makeatletter
\def\maxwidth{\ifdim\Gin@nat@width>\linewidth\linewidth\else\Gin@nat@width\fi}
\def\maxheight{\ifdim\Gin@nat@height>\textheight\textheight\else\Gin@nat@height\fi}
\makeatother
% Scale images if necessary, so that they will not overflow the page
% margins by default, and it is still possible to overwrite the defaults
% using explicit options in \includegraphics[width, height, ...]{}
\setkeys{Gin}{width=\maxwidth,height=\maxheight,keepaspectratio}
% Set default figure placement to htbp
\makeatletter
\def\fps@figure{htbp}
\makeatother
\setlength{\emergencystretch}{3em} % prevent overfull lines
\providecommand{\tightlist}{%
  \setlength{\itemsep}{0pt}\setlength{\parskip}{0pt}}
\setcounter{secnumdepth}{5}
% definitions for citeproc citations
\NewDocumentCommand\citeproctext{}{}
\NewDocumentCommand\citeproc{mm}{%
  \begingroup\def\citeproctext{#2}\cite{#1}\endgroup}
\makeatletter
 % allow citations to break across lines
 \let\@cite@ofmt\@firstofone
 % avoid brackets around text for \cite:
 \def\@biblabel#1{}
 \def\@cite#1#2{{#1\if@tempswa , #2\fi}}
\makeatother
\newlength{\cslhangindent}
\setlength{\cslhangindent}{1.5em}
\newlength{\csllabelwidth}
\setlength{\csllabelwidth}{3em}
\newenvironment{CSLReferences}[2] % #1 hanging-indent, #2 entry-spacing
 {\begin{list}{}{%
  \setlength{\itemindent}{0pt}
  \setlength{\leftmargin}{0pt}
  \setlength{\parsep}{0pt}
  % turn on hanging indent if param 1 is 1
  \ifodd #1
   \setlength{\leftmargin}{\cslhangindent}
   \setlength{\itemindent}{-1\cslhangindent}
  \fi
  % set entry spacing
  \setlength{\itemsep}{#2\baselineskip}}}
 {\end{list}}
\usepackage{calc}
\newcommand{\CSLBlock}[1]{\hfill\break\parbox[t]{\linewidth}{\strut\ignorespaces#1\strut}}
\newcommand{\CSLLeftMargin}[1]{\parbox[t]{\csllabelwidth}{\strut#1\strut}}
\newcommand{\CSLRightInline}[1]{\parbox[t]{\linewidth - \csllabelwidth}{\strut#1\strut}}
\newcommand{\CSLIndent}[1]{\hspace{\cslhangindent}#1}
\usepackage{setspace} \setstretch{1.15} \usepackage{float} \floatplacement{figure}{t}
\ifLuaTeX
  \usepackage{selnolig}  % disable illegal ligatures
\fi
\usepackage{bookmark}
\IfFileExists{xurl.sty}{\usepackage{xurl}}{} % add URL line breaks if available
\urlstyle{same}
\hypersetup{
  colorlinks=true,
  linkcolor={cyan},
  filecolor={Maroon},
  citecolor={Blue},
  urlcolor={cyan},
  pdfcreator={LaTeX via pandoc}}

\title{~\large Tournaments}
\author{\large Zach Culp \vspace{-1.1mm}\\
\normalsize Epi \vspace{-1mm}\\
\normalsize Oxford something at Emory \vspace{-1mm}\\
\normalsize Georgia? \vspace{-1mm}\\
\normalsize \href{mailto:ypu@something.edu}{\texttt{email}}
\vspace{-1mm}\\
\strut \\
\large Josie Peterburgs \vspace{-1.1mm}\\
\normalsize College of Education \vspace{-1mm}\\
\normalsize University of Texas at Austin \vspace{-1mm}\\
\normalsize Austin, TX 78712 \vspace{-1mm}\\
\normalsize \href{mailto:brian.mills@austin.utexas.edu}{\texttt{brian.mills@austin.utexas.edu}}
\vspace{-1mm}\\
\strut \\
\large Ryan McShane \vspace{-1.1mm}\\
\normalsize Department of Mathematics and Statistics \vspace{-1mm}\\
\normalsize Loyola University Chicago \vspace{-1mm}\\
\normalsize Chicago, IL 60660 \vspace{-1mm}\\
\normalsize \href{mailto:lderango@luc.edu}{\texttt{lderango@luc.edu}}
\vspace{-1mm}\\
\strut \\
\large Gregory J. Matthews \vspace{-1.1mm}\\
\normalsize Department of Mathematics and Statistics \vspace{-1mm}\\
\normalsize Center for Data Science and Consulting \vspace{-1mm}\\
\normalsize Loyola University Chicago \vspace{-1mm}\\
\normalsize Chicago, IL 60660 \vspace{-1mm}\\
\normalsize \href{mailto:gmatthews1@luc.edu}{\texttt{gmatthews1@luc.edu}}
\vspace{-1mm}}
\date{}

\begin{document}
\maketitle
\begin{abstract}
Here is the abstract.\\

\vspace{2mm} \textbar{} Keywords: survival analysis, hockey
\end{abstract}

\section{Introduction}\label{introduction}

Popkin et al. (\citeproc{ref-Popkin2023}{2023})

There are hundreds of different tournament structures that dictate how a
set of teams compete against each other to determine an overall ranking.
Different tournaments have unique strengths and weaknesses, affecting
how well they reflect the true rankings of the teams. The goal of a
tournament is to find the best teams, but how can we quantify how well a
tournament performs? In some instances, the tournament organizers only
award the overall winner, but other times, the top three or ten teams
are awarded. Because of this, organizers may prefer one structure over
another based on their needs. Tournament organizers must also take into
account factors like cost, timeliness, entertainment value, and fairness
when choosing a tournament structure.

\begin{verbatim}
To evaluate a tournament’s effectiveness, we propose a numerical metric that quantifies how accurately it orders teams based on their true rankings. Tournament results can be viewed as a “message” attempting to convey the true team rankings. However, various factors introduce noise, leading to information loss. We use principles from information theory to measure this information loss across multiple tournament simulations to assess the reliability of different formats.
\end{verbatim}

\section{Methods}\label{methods}

\section{Conclusion}\label{sec:conclusion}

\section*{Acknowledgements}\label{acknowledgements}
\addcontentsline{toc}{section}{Acknowledgements}

\section*{Supplementary Material}\label{supplementary-material}
\addcontentsline{toc}{section}{Supplementary Material}

\section*{References}\label{references}
\addcontentsline{toc}{section}{References}

\phantomsection\label{refs}
\begin{CSLReferences}{1}{0}
\bibitem[\citeproctext]{ref-Popkin2023}
Popkin, Charles A., Cole R. Morrissette, Thomas A. Fortney, Kyle L.
McCormick, Prakash Gorroochurn, and Michael J. Stuart. 2023. {``Fighting
and Penalty Minutes Associated with Long-Term Mortality Among National
Hockey League Players, 1967 to 2022.''} \emph{JAMA Network Open} 6 (5):
e2311308--8.

\end{CSLReferences}

\end{document}
